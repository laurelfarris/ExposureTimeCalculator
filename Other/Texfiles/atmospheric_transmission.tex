\documentclass[12]{article}
\usepackage[margin=1in]{geometry}
\usepackage{mdwlist}
\setlength{\parindent}{0em}
\setlength{\parskip}{0.75em}

\title{Documentation for atmospheric transmission code}
\author{Laurel Farris}
\date{\today}


\begin{document}
\vspace{-2in}
\maketitle

\section*{Intro}
This document describes the thought processes, required math, etc.\
for the code \verb|atmospheric_transmission.py|.

\section*{Math}

The following is the math carried out to get the atmospheric transmission,
$a_{\lambda}$ in terms of the user input:
\begin{itemize*}
    \item object magnitude: $M_{obj}$
    \item airmass: $X$
\end{itemize*}
$M_{net}$ is the ``net'' transmitted magnitude observed from the ground.
Equations

$$ M_{net} = M_{obj} + kX $$
$$ M_{net} - M_{obj} = -2.5\log\left(\frac{F_{net}}{F_{obj}}\right) $$
$$ M_{net} - M_{obj} = -2.5\log\left(a\right) $$
$$ a = 10^{\wedge}\left(\frac{M_{net}-M_{obj}}{-2.5}\right) $$
$$ a = 10^{\wedge}\left(\frac{kX}{-2.5}\right) $$

\end{document}
